\documentclass{article}

\usepackage{./goban/goban}

\begin{document}
  % \begin{figure}[ht]
  %   \begin{center}
  %     \begin{tikzpicture}
  %       \goGrid[board dimension = 5,
  %               board size      = 9,
  %               coords          = true]

  %       % \parseSgfB{\sgfA}
  %       \stone[color = black,
  %              x     = 3,
  %              y     = 3,]
  %       \stone[color = white,
  %              x     = 4,
  %              y     = 4,]
  %       \move[color = white,
  %              x    = 5,
  %              y    = 5,]

  %       \move[color = black,
  %              x    = 6,
  %              y    = 6,]

  %       % \drawMove[color = black,
  %       %           x     = 5,
  %       %           y     = 5,]

  %       % \drawStoneFromSgfCoords{black}{ab}
  %       % \drawMoveFromSgfCoords{black}{cd}

  %     \end{tikzpicture}
  %     \caption{Goban 1}\label{my_goban_1}
  %   \end{center}
  % \end{figure}

  \begin{figure}[ht]
    \begin{center}
      \begin{goban}[board dimension = 10,
                    board size      = 19,
                    coords          = true,
                    scale           = 1]


        \drawStoneFromSgfCoords{black}{ab}

        % \drawMoveFromSgfCoords{white}{de}
        % \drawMoveFromSgfCoords{black}{cd}

        % \drawStonesFromSgfCoords{{black, ab}, {white, cd}}
        % \drawMovesFromSgfCoords{{black, ab}, {white, cd}}{false}
        % \parseSgf{\sgfA}

      \end{goban}
      \caption{Goban 2}\label{my_goban_2}
    \end{center}
  \end{figure}

  % \parseSgfB{\sgfA}
\end{document}