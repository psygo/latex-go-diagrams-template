% \documentclass{article}

% \usepackage{tikz}

% \newlength{\step}

% \newcommand{\parsesgf}[1]{
% }

% \begin{document}
%   \parsesgf{abc}

%   \begin{tikzpicture}
%     \setlength{\step}{\dimexpr 10cm / 18 \relax}

%     \draw[step=\step] (0, 0) grid (10, 10);

%     \draw[draw = white, fill = black, line width = 0.1mm]
%       (2 * 10cm / 18, 3 * 10cm / 18)
%       circle [radius = 0.2575cm]
%       node[color = white] {1};
%     \draw[draw = black, fill = white, line width = 0.1mm]
%       (3 * 10cm / 18, 3 * 10cm / 18)
%       circle [radius = 0.2575cm]
%       node[color = black] {2};
%     \draw[draw = white, fill = black, line width = 0.1mm]
%       (4 * 10cm / 18, 3 * 10cm / 18)
%       circle [radius = 0.2575cm]
%       node[color = white] {3};
%   \end{tikzpicture}
% \end{document}

\documentclass[border=10pt]{standalone}
\usepackage{tikz}

% Parameters
%
% 1: dimension (in cm)
% 2: board size (square)
% 
% Example: A 19x19 board with size 10cm x 10cm: `\gogrid{10}{19}'
\newcommand{\goGrid}[2]{
  \pgfmathsetmacro{\step}{#1 / (#2 - 1)} % chktex 1
  \draw[step=\step] (0, 0) grid (#1, #1);
  \drawHoshis{#1}{#2}{\step}
}

% Parameters
%
% 1: dimension (in cm)
% 2: board size (square)
% 3: step
% 
% Example: A 19x19 board with size 10cm x 10cm: `\drawHoshis{10}{19}{\step}'
\newcommand{\drawHoshis}[3]{
  \ifnum#2=9\relax 
    \foreach \sloc in {{3, 3}, {3, 7}, {7, 3}, {7, 7}, {5, 5}}{
      \pgfmathsetmacro{\hoshiCoordX}{#3 * ({\sloc}[0] - 1)}
      \pgfmathsetmacro{\hoshiCoordY}{#3 * ({\sloc}[1] - 1)}
      \filldraw (\hoshiCoordX, \hoshiCoordY)
        circle[radius={#3 / 10}];
    }
  \fi
}

\begin{document}
  \begin{tikzpicture}
    \goGrid{10}{9}
  \end{tikzpicture}
\end{document}